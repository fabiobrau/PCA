\documentclass[10pt]{beamer}
\usepackage{bbm}
\usepackage{graphicx}
\renewcommand\appendixname{Appendix}
\usepackage{xcolor}
\usepackage{tikz}
\colorlet{rred}{red!80!black}
\colorlet{ggreen}{green!80!black}




\usetheme[progressbar=foot]{metropolis}
\usepackage{appendixnumberbeamer}
\setbeamercovered{dynamic}

\usepackage{booktabs}
\usepackage[scale=2]{ccicons}

\usepackage{pgfplots}
\usepgfplotslibrary{dateplot}
\setbeamertemplate{caption}{\raggedright\insertcaption\par}
\setlength{\abovecaptionskip}{-10pt plus 0pt minus 0pt}

\usepackage{xspace}
\newcommand{\themename}{\textbf{\textsc{metropolis}}\xspace}
\theoremstyle{definition}
\newtheorem{defn}{Definition}

% math symbols
\newcommand{\R}{\mathbb{R}}
\newcommand{\N}{\mathbb{N}}
\newcommand{\Z}{\mathbb{Z}}
\newcommand{\1}{\mathbbm{1}}
\newcommand{\XX}{\mathcal{X}}
\newcommand{\TT}{\mathcal{T}}
\newcommand{\YY}{\mathcal{Y}}

\title{An introduction to PCA}
\subtitle{Weekly AI pills}
\date{2020-10-16}
\author{Fabio Brau.}
\institute{SSSA, Emerging Digital Technologies, Pisa.}
% \titlegraphic{\hfill\includegraphics[height=1.5cm]{logo.pdf}}

\usebackgroundtemplate{%
    \begin{picture}(300,271)
      \hspace{11.2cm}
       \includegraphics[scale=0.1]{pic/logoretis_noname.png}
   \end{picture}}

\begin{document}
{\usebackgroundtemplate{%
    \begin{picture}(300,265)
      \hspace{0.9cm}
       \includegraphics[scale=0.5]{pic/tecip_logo.png}
       \hspace{0.5cm}
       \includegraphics[scale=0.21]{pic/logoretis_320.png}
   \end{picture}}%
\maketitle
}
\begin{frame}{Summary}
  \begin{itemize}
    \item The aim of Principal Component Analysis
    \item Derivation
      \begin{enumerate}
          \item A Geometrical idea
          \item A statistical Derivation
          \item Singolar Value Decomposition
      \end{enumerate}
    \item PCA from Encoder Decoder NN
    \item Dummy examples
  \end{itemize}
\end{frame}
\begin{frame}{A Geometrical Idea}
  Let $X \in \R^{N\times n}$ be a dataset of $N$
  {\bf observation} within $n$ {\bf variables}. 
  \begin{equation}
    X =
    \begin{bmatrix}
      \, & x_1 ^ T &\,\\
      \, & \vdots &\,\\
      \, & x_N ^ T &\,\\
    \end{bmatrix}
    =
    \begin{bmatrix}
      & & & \\
      x^{(1)} &\vline& \cdots& \vline& x^{(n)}\\
      & & & \\
    \end{bmatrix}
    \label{Geometrical view}
  \end{equation}
  {\bf Notations:}
  \begin{itemize}
    \item $x_i\in\R^n$ represents a single {\bf observation}, i.e a {\bf
      sample} in the feature space.
    \item $x^{(i)}\in\R^N$ represents the single {\bf variable}, i.e a {\bf
      column} of the dataset.
    \item The object $\1_n\in\R^n$ is the unitary columnar vector of length
      $n$ $\1_n=[1,\cdots,1]$.
  \end{itemize}
\end{frame}
\begin{frame}
  
\end{frame}<++>
\end{document}

